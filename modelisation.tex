\documentclass[12pt,a4paper]{article}
\usepackage[utf8]{inputenc}	
\usepackage[francais]{babel}
\usepackage[T1]{fontenc}
\usepackage[left=2cm,right=2cm,top=2cm,bottom=2cm]{geometry}
\usepackage{graphicx}
\usepackage{lmodern}
\usepackage{url}
\usepackage{amsmath}
\usepackage{amsfonts}
\usepackage{amssymb}
\usepackage[lf]{venturis}

\title{Modélisation}
\author{\textsc{Charlie Pauvert} \& \textsc{Guillaume Reboul}}

\begin{document}

\section{Introduction}
\subsection{Origine des données}
Les données utilisée dans notre projet sont des données issue du génome complet d'une souche d'Archée : Pyrococcus furiosus DSM 3638.
Nous avons choisi ce génome car, d'une part, essayer un modèle de Markov Caché sur des génomes peu communs comme les Archées afin de modéliser des CDS nous semble interessant et, d'autre part, ce génome est très connu sans l'être. En effet, c'est de cette Archée anaérobie et hyperthermophile qu'est issue la Pfu Polymérase, enzyme utilisée dans les PCR en routine et connue pour comettre moins d'erreurs que son homologue : la Taq polymérase.
\\ \indent
Le génome, qui date de 2002, a été téléchargé depuis le site du NCBI en format fasta. Il comporte 1 908 256 nucléotides et son taux de GC est de 40,80\%. Selon le site du NCBI, le génome posséde 2088 gène pour 1989 protéines.
\subsection{Les interêts du projet}
L'intêret de ce projet et d'utiliser les modéles de Markov et de Markov caché à des fins bioinformatiques et plus précisement à la prédiction de gènes. En effet, l'alternance de régions codantes et non codantes dans un génome procaryote comme celui de Pyrococcus furiosus peut être modéliser par un modèle de markov caché à 2 états caché.

\section{Modélisation avec une chaîne de Markov}
\subsection{Présentation de l'objectif}
Le premier objectif de ce projet est de modéliser notre séquence du génome de Pyrococcus furiosus DSM 3638 à l'aide de modèles de type markovien.
Pour cela, nous avons utilisé trois types d'approches afin de déterminer l'ordre du modèle de markov qui représente le mieux le génome.
Ces trois approches sont : le rapport de vraisemblance, le critère AIC et le critère BIC.
\subsection{Résultats}
Nous avons donc testé ces 3 critères pour des ordres allant de 0 à 7 afin de définir l'ordre pour lequel le modèle de Markov est le plus explicatif du génome.
\begin{table}[!h]
\centering
\begin{tabular}{l|r|r|r}
\hline
Ordre & pvalue & AIC & BIC\\
\hline
0 & 0 & -875042.9 & -874993.1\\
\hline
1 & 0 & -901953.7 & -901754.3\\
\hline
2 & 0 & -909306.3 & -908508.8\\
\hline
3 & 0 & -913593.3 & -910403.1\\
\hline
4 & 0 & -916255.0 & -903494.2\\
\hline
5 & 1 & -914946.1 & NA\\
\hline
6 & NA & NA & NA\\
\hline
7 & NA & NA & NA\\
\hline
\end{tabular}
\end{table}
\paragraph{Le test du rapport de vraisemblance}
Comme on peut le voir dans le tableau, nous avons réalisé le test du rapport de vraisemblance pour les ordres allant de 0 à 7.
On peut observer que pour les ordres allant de 0 à 4 compris, les p-values résultantes sont de 0. C'est à dire qu'elles sont inférieures à notre risque de première espèce qui est de 5\%. On rejette donc l'hypothèse H0 qui correspond au fait que le modèle le plus explicatif est le modèle d'ordre <<Ordre>> et non d'ordre <<Ordre+1>>. Ensuite, à l'ordre 5, on observe une p-value de 1 ce qui est nettement supérieur au risque de première espèce utilisés lors de ce test. L'hypothèse H1 est donc rejettée, le modèle d'Ordre 6 n'est pas plus explicatif que le modèle d'ordre 5.
\\ \indent
On arrete donc les calculs et on peut conclure que le test de rapport de vraisemblance nous suggére que notre génome est modélisable de la meilleure des façon avec un modèle de Markov d'ordre 5.
\paragraph{Le critère AIC}
Les résultats issus de nos calculs des critères AIC sont visibles dans le tableau dans la colone AIC. Les valeurs sont celles obtenues et on peut observer que les valeurs diminues jusqu'à l'AIC d'ordre 4 (-916255.0). On observe pour l'ordre 5 une légère augmentation (-914946.1) par rapport à l'ordre 4 donc l'algorithme c'est arrêter automatiquement. En effet, le but est de trouver l'ordre qui minimise la valeur d'AIC. Afin de ne pas calculer inutilement, nous avons codé la fonction de sorte qu'elle s'arrete quand une hausse est observée pour la valeurs d'AIC de l'orbre k par rapport à la valeur obtenue pour k-1.
\\ \indent
La conclusion ici est que le critère AIC nous oriente vers un modèle de Markov d'ordre 4 pour modéliser au mieux notre génome.
\paragraph{Le critère BIC}
Le critére BIC est représenté dans le tableau pas la colone BIC. On peut observer une décroissance des valeurs du critère BIC entre les ordres 0 et 3 (de -874993.1 à -910403.1). On note ensuite une légère augmentation au niveau de l'ordre 4 (-903494.2) par rapport a la valeur BIC de l'ordre 3 (-910403.1). Comme pour le critère AIC, car c'est la même fonction, les calculs s'arrêtent lorsque le critère est minimisé.
\\ \indent
On peut donc conclure que l'ordre qui minimise le critère BIC est l'ordre 3. Le critère BIC nous oriente donc vers le choix d'un modéle de Markov d'ordre 3 pour modéliser notre jeu de données.

\section{Détection des régions homogènes au long du génome}
\subsection{Présentation de l'objectif}
Ici, l'objectif est de trouver l'alternance de région codantes et non codantes grâce aux chaînes de Markoc cachées. En effet, en appliquant un certain modèle et en procédant à un apprentissage, on souhaiterai prédire correctement l'alternance entre les régions codantes et non codantes de Pyrococcus furiosus.
\\ \indent
Tout d'abord on notera que l'on suppose être dans un modèle de type Markov caché M1-M0. Une première étape sera d'estimer les paramètres de ce modèle à l'aide de l'algorithme EM. Une seconde étape consiste à implémenter l'algorithme de viterbi afin de determiner la suite des états cachés au long du génome de notre Archée. Enfin on comparera le nombre de prédiciton, la taille moyenne des CDS/gènes et nos autres résultats à des programmes de recherches de CDS librement disponible sur internet.
\subsection{Résultats}

\paragraph{Le rapport de vraisemblance}

%%%%%%%%%%%%%%%%%%%%%%%%%%%%%%%%%%%%%%%%%%%%%%%%%%%%%

/bin/bash: :q : commande introuvable
\section{Notation }

On souhaite modéliser la séquence de l'organisme choisi avec des modèles de chaînes de Markov. 
L'enchaînement des acides nucléiques dans les séquences est rarement indépendant (pression de sélection).

On va examiner la pertinence de différents ordres de chaînes de Markov. 


\subsection{Modéle}

On défini les $X_1X_2, \dots, X_i, \dots, X_n$ comme les $n$ variables aléatoires décrivant l'enchaînement des lettres dans une séquence.
Une chaîne de Markov d'ordre $k$ est défini par : 

\begin{itemize}
\item un alphabet est noté $\mathcal{A} = \{A,C,G,T\}$.
\item une loi initiale $\mu_0$
\item une matrice de transition $\pi$ avec $k$ colonnes et $|\mathcal{A}|^k$ lignes.
\end{itemize}

Selon la propriété de Markov, on a  : 

\[
\mathbb{P}(X_{i+1} = \beta | X_i = \alpha ) 
\]


\subsection{Résultats}

\paragraph{Le rapport de vraisemblance}



Un modèle de chaîne de markov caché permettrait peut être d'accéder à une meilleure modélisation.

\end{document}
