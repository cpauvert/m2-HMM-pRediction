\documentclass[12pt,a4paper]{article}
\usepackage[utf8]{inputenc}	
\usepackage[francais]{babel}
\usepackage[T1]{fontenc}
\usepackage[left=2cm,right=2cm,top=2cm,bottom=2cm]{geometry}
\usepackage{graphicx}
\usepackage{lmodern}
\usepackage{url}
\usepackage{amsmath}
\usepackage{amsfonts}
\usepackage{amssymb}

\title{Modélisation}
\author{\textsc{Charlie Pauvert} \& \textsc{Guillaume Reboul}}

\begin{document}

/bin/bash: :q : commande introuvable
\section{Notation }

On souhaite modéliser la séquence de l'organisme choisi avec des modèles de chaînes de Markov. 
L'enchaînement des acides nucléiques dans les séquences est rarement indépendant (pression de sélection).

On va examiner la pertinence de différents ordres de chaînes de Markov. 


\subsection{Modéle}

On défini les $X_1X_2, \dots, X_i, \dots, X_n$ comme les $n$ variables aléatoires décrivant l'enchaînement des lettres dans une séquence.
Une chaîne de Markov d'ordre $k$ est défini par : 

\begin{itemize}
\item un alphabet est noté $\mathcal{A} = \{A,C,G,T\}$.
\item une loi initiale $\mu_0$
\item une matrice de transition $\pi$ avec $k$ colonnes et $|\mathcal{A}|^k$ lignes.
\end{itemize}

Selon la propriété de Markov, on a  : 

\[
\mathbb{P}(X_{i+1} = \beta | X_i = \alpha ) 
\]

\end{document}
